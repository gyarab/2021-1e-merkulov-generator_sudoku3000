\chapter{Úvod}
Sudoku je logická hra s číslicemi. Cílem hry v základní podobě je doplnit chybějící cifry 1-9 v zadané, zčásti vyplněné čtvercové mřížce s 9×9 poli. V tabulce jsou zvýrazněny 4 příčky vymezující 9 čtverců (3×3). K předem vyplněným číslicím je třeba doplnit další číslice tak, aby platilo, že v každém řádku, v každém sloupci a v každém z 9 čtverců 3x3 jsou použity vždy všechny číslice 1-9 a každá právě jednou. Tento dokument se zabívá generátorem náhodného řešitelného sudoku. Generátor dokáže vygenerovat sudoku různé velikosti a různé obtížnosti. Program je napsán v programovacím jazyce Java a grafické okno se vykresluje pomocí knihovny Swing. K řešení problému jsem přistoupil, tak že nejdříve by se vyplnila mřížka sudoku všemi čísly a potom by se náhodně vymazali čísla z mříšky, abychom jsme dostali hotový hlavolam. Při ukázce kódů budu dávat jako příklad kódy ke generování sudoku 9x9 pro jiné velikosti by stačilo zaměnit hodnotu 9 jinou příslušnou hodnotou (4, 6).\\
\\
\begin{Large}
\textbf{Zadání}\\
\\
\end{Large}
Zadání mé ročníkové práce bylo:\\
\\
1. Vygenerování náhodného řešitelného sudoku\\
2. Vygenerování různé velikosti sudoku\\
3. Vygenerování sudoku různé obtížnosti